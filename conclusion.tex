\documentclass[paper.tex]{subfiles}

\begin{document}




\section{Implications for ODEs in general}

Very brief exposition on section 5 in a knotty ode discussing what the universality of $\V$ implies in general. Proving universality is reduced to showing that a template can be renormalized to $\V$


\section{Miscellaneous discussions and interesting open questions}
\label{sec:misc}

The last theorem of Section 2.4 hints at a requirement for two templates to be considered equal: given templates $A$ and $B$, $A$ and $B$ are equal if $A \subset B$ and $B \subset A$. If each template may be contained within the other, then all knots in $A$ must be contained in $B$ as well, and all knots in $B$ must be contained in $A$ as well.

Discussion of $A_\tau$. How much information does it tell us? How hard is to compute from a given ODE (very hard)? What additional information should we keep track off (ie crossings) to make it relatively easy to compute
the link of knots contained in a given template?




\end{document}
