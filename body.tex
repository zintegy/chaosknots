\documentclass[paper.tex]{subfiles}

\begin{document}


\section{Proof that $\mathcal{V}$ is a universal template}

In this section we outline the proof given in full detail in~\cite{Ghrist1996} that $\V$ (shown in figure~\ref{fig:universal}) is a universal template. Details will inevitably be omitted but our goal in this exposition is to straddle the line between
the full proof and the very high-level outline given in~\cite{knottyode}, giving a concise and readable exposition that gets at the essence of the proof.

The essential idea is to find a special set of templates which have a special property that forces them to support all braids as orbits, and then to show that these templates can be found in $\V$. Since all knots and links
can be realized as a braid, this immediately implies that $\V$ is universal. The last step of showing that the special braid-supporting templates can found in $\V$ is the most difficult by far and where we will have to do the
most hand-waving. Nonetheless, we will try to capture the essence.

\begin{figure}[h]
  % not sure where to put this, but not here
  \centering
  \includegraphics[width=0.5\textwidth]{universal.png}
  \caption{The universal template, reproduced from~\cite{knottyode}}\label{fig:universal}
\end{figure}

\subsection{Braids and the theorem of Alexander}

Recall that a braid is a collection of $P$ disjoint simple closed curves in a standardly embedded torus such that every cross-section of the torus intersects the braid in exactly $P$ points.
Braids have a natural identification as a permutation on $P$ elements, which is the property we exploit. In a landmark paper~\cite{Alexander1923}, Alexander proved the following theorem which provides the crucial connection
between braids and links.

% we exploit it since each permutation can be written as the product of exchanges

\begin{thm}[Alexander 1923]
  Each knot or link is isotopic to some closed braid on $P$ strands for some $P$
\end{thm}

\subsection{The templates $\W_q$}

The family of templates $\set{\W_q}$ shown in figure~\ref{fig:w_q} are those referred to earlier as the braid-supporting templates. $\W_1$ is identically $\V$ and increasing $q$ by one has the effect of adding
two \emph{ears} to one side. The property that successive ears alternate in `sign' is what makes this family of templates so useful, as this property makes proving that they support all braids delightfully simple.

\begin{figure}[h]
  \centering
  \includegraphics[width=0.8\textwidth]{w_q.png}
  \caption{The templates $W_q$, reproduced from~\cite{knottyode}}\label{fig:w_q}
\end{figure}


\begin{lemma}[Ghrist 1996]
  An isotopic copy of any closed braid exists as a set of periodic orbits on some $W_q$ for sufficently large $q$.
\end{lemma}

\begin{proof}
  Braids have a natural connection to the symmetric group $S_n$. A single braid can be identified with an element $s \in S_n$ (for the elements of the symmetric group are permutations) and the concatenation of succesive braids
  mirrors the group operation on $S_n$. In this way, the concatenation of alternative positive and negative ears on $\W_q$ mimics the group operation on $S_n$. We seek to show that a generating set for $S_n$ can be `fit' onto a
  finite concatenation of alternating ears, as occurs in $\W_q$.

  $S_n$ is generated by $\set{\sigma_1, \ldots \sigma_{n - 1}}$ where $\sigma_i$ transposes the ith and (i+1)th strands of the braid. The $\sigma_i$ are subject to the relations $\sigma_i \sigma_j = \sigma_j \sigma_i$
  for $\abs{i - j} > 1$ and $\sigma_i \sigma_j \sigma_i = \sigma_j \sigma_i \sigma_j$ for $\abs{i - j} = 1$.

  Figure~\ref{fig:earexchange} illustrates how to put the word $\sigma_1 \sigma_2, \ldot \sigma_k$ on a positive ear and $\sigma_1^{-1} \sigma_2^{-1} \ldots \sigma_k^{-1}$ on a negative ear. In particular, we explicitly get
  $\sigma_1$ and $\sigma_1^{-1}$ on respectively positive and negative ears. We proceed by induction, asssuming that we can find the generators $\sigma_1, \sigma_1^{-1}, \ldots \sigma_k, \sigma_k^{-1}$ on a finite sequence
  of ears. We can then construct $\sigma_{k+1}$ and $\sigma_{k+1}^{-1}$ by concatenating additional ears as follows


  \begin{align}
    \sigma_{k+1} &= (\sigma_k^{-1}) \ldots (\sigma_2^{-1})(\sigma_1^{-1})(\sigma_1 \sigma_2 \ldots \sigma_{k+1})
    \sigma_{k+1}^{-1} &= (\sigma_k) \ldots(\sigma_2) (\sigma_1) (\sigma_1^{-1} \sigma_2^{-1} \sigma_{k+1}^{-1})
  \end{align}

  Since we have shown how to construct the words $\sigma_1 \sigma_2 \ldots \sigma_{k+1}$ and $\sigma_1^{-1} \sigma_2^{-1} \sigma_{k+1}^{-1}$ on a single ear, this shows by induction that we can fit the generators for $S_n$ for
  any $n$ on a finite sequence of alternating ears.

  %there's an additional note in the proof given in Ghrist which I don't understand
\end{proof}



\begin{figure}[h]
  \centering
  \includegraphics[width=0.8\textwidth]{ear_exchange.png}
  \caption{Fitting a generating set for $S_n$ on the ears of $\W_q$. Figure reproduced from~\cite{Ghrist1996}}\label{fig:earexchange}
\end{figure}






\subsection{Construct $\W_{q+1}$ from $\W_q$}

Start with renormalized $\W_1 \in \V$, and append a pair of ears to\dots

\subsection{Find $\W_q \subset \V$ for all $q$}

This is very tricky and will have to be very loosely sketched

This paper is referenced to prove this fact. (requires defining subtemplates; maybe that could be done in an earlier section)
https://www.math.upenn.edu/~ghrist/preprints/eraams.pdf

\end{document}
